\documentclass[9pt,twocolumn,twoside]{article}
\usepackage{amsmath}
\usepackage{amssymb}
\usepackage{bbm}
\usepackage{mathtools}

\begin{document}

\title{Lattice Gauge Theories with Discrete Gauge Groups}
\author{Hunter Swan}

\maketitle
\begin{abstract}
I describe the formulation of lattice gauge theories for a variety of discrete gauge groups, including the cyclic groups $Z_N$, the Klein 4 group $K_4$, and the quaternion group $Q$.  Using Monte Carlo methods, I map the phase diagrams for these theories in various dimensions and identify phase transitions.
\end{abstract}

\section{Introduction}
Quantum field theory is \textit{hard}.  The playing field for quantum field theory (QFT) is a massively infinite dimensional Hilbert space, and the players themselves are operator valued functions of spacetime coordinates with infinitely many quantum degrees of freedom.  The dynamics are represented by functional integrals---infinite dimensional integrals over the space of fields---which by their nature are impossible to compute in any concrete way (e.g. on a computer via some sort of quadrature), and usually don't have the decency to converge anyway.  As if to add insult to injury, even in those cases where one can evaluate a quantity in QFT, it often turns out to be $\infty$ (perhaps not a surprise, given the number of infinities encountered already in this paragraph).  The situation is so desperate, mathematicians have put up a million dollar bounty for the first person to tame QFT.

Dispite its aesthetic failings, QFT is nevertheless \textit{right}, insofar as we can tell.  That is to say, QFT still manages to make predictions, and those predictions are in exquisite agreement with reality (for example, the theory of quantum electrodynamics, QED, has been tested to 15 decimal places with no deviations so far).  So regardless of our possible prejudices against the mathematical or computational character of QFT, we pretty much have to live with it.  The burden therefore seems to be on us, the examiners, to reconcile ourselves to QFT, rather than holding out for QFT to reconcile itself to our idealogies.  

Enter \textbf{lattice quantum field theory} (LQFT).  This is the same as regular QFT (which we shall refer to as continuum QFT or CQFT for the purposes of differentiating the two), but with spacetime discretized on a lattice.  This seems innocent enough, and in fact it \textit{is} far more innocent than its reprobate cousin CQFT: LQFT is perfectly well-defined mathematically, and even numerically tractable.  What makes LQFT even more attractive is that, by taking appropriate continuum limits, it can provide answers to questions of CQFT!

So the first reason we should care about LQFT is that it provides a palatable and productive way to approach continuum QFT.  However, even apart from continuum limits, LQFT has interesting structure in its own right.  Some lattice quantum field theories turn out to be exactly equivalent to systems arising from statistical mechanics, and as such they display the same sorts of phase transitions and critical behavior characteristic of the latter systems.  LQFT thus provides a new perspective on statistical mechanics, helping elucidate the structure of systems with many degrees of freedom.  This is roughly the thrust of the present paper.  My main goal is to look at a variety of lattice gauge theories (particular kinds of LQFT, to be defined below) and examine how the phase structure of the theory depends on its intrinsic properties (namely, its gauge group and the dimension of the ambient lattice).  This has been shown to be an interesting paradigm before \cite{Creutz,DAdda}, and I'll revisit some of this work below.  I'll also look at some other lattice gauge theories, showing that they ...

\section{Definitions and QFT background}
Normally, one thinks of quantum field theory as residing in Minkowski space, i.e. $\mathbb{R}^4$ with metric $ds^2 = -dt^2 + dx^2 + dy^2 + dz^2$.  Physicists like to say that this metric puts time and space ``on equal footing''.  The argument goes something like this: ``Blah blah blah blah special relativity blah blah blah.''  Of course this argument is total hogwash: Any preschooler can tell you that the time coordinate in this metric gets a minus sign, while the space coordinates have plus signs.  There is a cure, though!  If we set $\tau := \pm i t$, then we can rewrite the metric as $ds^2 = d\tau^2 + dx^2 + dy^2 + dz^2$.  Now the ``imaginary time'' coordinate $\tau$ really is on equal footing with the spatial coordinates.  What we've done essentially is to embed $\mathbb{R}^4$ into $\mathbb{C}\times \mathbb{R}^3$ and then restrict to the imaginary axis of $\mathbb{C}$.  Geometrically we haven't done anything fancy.  However, once we start considering functions of $\mathbb{R}^4$ (fields), in order for the corresponding rotation of the functions to be unambiguous we will need to ensure that they all are analytic in time, so that we can uniquely extend them from the reals to the complex plane.  Fortunately, physicists are confident that this can be done in all the cases we care about.  The proof goes as follows: If we were to find a function which was non-analytic so that we could not rotate it in this way, then by definition, we no longer care about it.  QED

So assume that we can do this rotation in the complex plane (this is called a Wick rotation).  After this section we will always work in this Wick rotated space.  The rationale for doing this is partly, as mentioned above, to symmetrize the roles of time and space coordinates, but another reason stems from the effect this has on dynamics of a quantum field theory.  Loosely speaking, putting an $i$ in a function argument turns oscillatory behavior into exponentially growing/decaying behavior and vice versa (as is the case for e.g. $\sin(\cdot)$, $\cos(\cdot)$, and $\exp(\cdot)$).  Oscillating functions are bad for integrals: They take forever to converge (if they converge at all), and the eventual value of the integral depends on lots of cancellation of regions with different phase.  By contrast, exponentially decaying functions exhibit no such pathologies---they are a complete joy to integrate.  Dynamics of quantum field theory in Minkowski space is described by an oscillatory integral.  After Wick rotation, this oscillatory integral becomes a much better behaved integral with an exponentially damped integrand.  This makes it far preferable for numerical work (not to mention rigorous mathematical work, where the Wick rotated integral has a fully satisfactory definition, whereas the oscillatory integral has thus far only been established precisely in a few cases).  We will see how this comes about explicitly in a little bit, but first we lay down some terminology:

%About the only function we will ever need to actually perform this rotation on is the action.  We'll do that in just a bit, after we lay down some definitions:

\begin{itemize}
\item A \textit{continuum field} is a function on $\mathbb{R}^n$.  Continuum fields are typically either scalar valued (classical fields) or operator valued (quantum fields).
\item An \textit{action} is a real-valued function of one or more continuum fields.
\item A \textit{continuum field theory} is a collection of continuum fields over the same space, together with an action on those fields and an additional set of prioritized functions of the fields called \textit{observables}.
\item A \textit{lattice}, for our purposes, is a graph with vertices $\mathbb{Z}_{m_1}\times\mathbb{Z}_{m_2}\times \cdots \times \mathbb{Z}_{m_n}$ and edges connecting two vertices if and only if those vertices' coordinates differ by exactly one unit (mod $m_j$) in exactly one direction $j$.  We will speak of an edge being ``in direction $j$'' if the vertices it connects are adjacent in that direction.  The (mod $m_j$) means we're making everything periodic from the get-go.  The lattice has associated with it a \textit{lattice spacing} $a_0$, which describes the distance between lattice points when embedded into $\mathbb{R}^n$.  We'll only need one lattice at a time in this paper, which we will call $L$, and since we won't be taking any continuum limits we will generally forget about the lattice spacing $a_0$.  We will use the following notation for talking about lattices: $V(L)$ will denote the vertices of $L$; $E(L)$ will denote the edges of $L$; and for any edge $e\in E(L)$ in direction $j$ with vertices $v,w\in V(L)$, if $v_j-w_j\equiv 1$ (mod $m_j$) we set $e_- = w$ and $e_+ = v$, i.e. $e_-$ is the ``lower'' vertex of $e$ and $e_+$ is the ``upper'' vertex.  (This definition of $e_-$ and $e_+$ is ambiguous when $m_j\leq 2$, so we will always require that $L$ have more than 2 points in all directions.)
\item A \textit{lattice field} is a function of a lattice's edges or vertices.  Most of the fields we will be interested in for this paper will take values in a group (the local symmetry group of the theory).  We use the term \textit{field} to refer to either a lattice field or continuum field, depending on context.
\item \textit{Lattice action} and \textit{lattice field theory} are defined analogously to the continuum case. 
\item A \textit{gauge group} $\mathcal{G}$ is a set of fields $T$ taking values in a group $G$, called the local symmetry group.  We require that $\mathcal{G}$ be closed under pointwise multiplication of such fields $T$.  As the name suggests, $\mathcal{G}$ is itself a group, with the group operation given by pointwise multiplication of fields.  In the continuum case, all the fields $T\in\mathcal{G}$ should be smooth.
\item A \textit{gauge theory} (lattice or continuum) is a field theory together with a gauge group $\mathcal{G}$ and a group action of $\mathcal{G}$ on the fields of the field theory, such that the action and all observables are invariant under the group action.  The group action by any particular element $T \in G$ is called a \textit{gauge transformation}.  Sometimes we also refer to $T$ itself as a gauge transformation.
\end{itemize}

Lattice gauge theories are the objects of interest in this paper.  I have so far provided no motivation for why gauge theories are more interesting than any other field theory.  I think the best explanation is simply that they exist: quantum electrodynamics, the electroweak interaction, and quantum chromodynamics are all gauge field theories.  I don't know a principled reason \textit{why} these theories should be gauge theories, but regardless, they are, and ergo we study gauge theories.

Many of the commonly studied lattice gauge theories are inherited from the continuum field theories of the standard model (especially quantum chromodynamics).  It turns out there is a reasonably algorithmic way to turn continuum field theories into lattice theories, as follows:
\begin{enumerate}
\item Wick rotate the action (and any relevant observables). 
\item Discretize space onto a lattice, thereby making all continuum fields lattice fields.
\item Choose a lattice action which reduces to the Wick rotated continuum action as the lattice spacing vanishes.
\end{enumerate}

This prescription is not totally algorithmic, because (for one thing) there can be many lattice actions that reduce to the same continuum action.  Moreover, there are lots of potentially interesting lattice field theories which cannot be obtained in this manner---an issue we will take up in the next section.

As an example of how this works in practice, consider a free scalar field in Minkowski space, $\phi:\mathbb{R}^n\rightarrow \mathbb{R}$, with action defined as 
\[S[\phi] := \int d^nx \frac{1}{2}(\partial_{\mu}\phi) (\partial^{\mu}\phi) - \frac{1}{2}m^2\phi^2.\]
After Wick rotating $t$ to $-it$, the $(\partial_0\phi)^2$ term picks up a $-1$ and the measure $dt$ picks up a $-i$, so the action becomes
\[i \int d^nx \frac{1}{2}(\partial_0\phi)^2 + \frac{1}{2}(\nabla\phi)^2 + \frac{1}{2}m^2\phi^2 =: i S_E[\phi].\]
$S_E[\phi]$ is called the Euclidean action.  It can be discretized onto a lattice in an obvious way:
\[ S_E[\phi] \rightarrow \sum_{x\in V(L)} \left( \frac{1}{2}m^2\phi(x)^2 + \frac{1}{2} \sum_{\mu=0}^n (\nabla_{\mu} \phi(x))^2 \right),\]
where we have introduced the notation $\nabla_{\mu} \phi(x) := (\phi(x+\hat{e}_{\mu}) - \phi(x))/a_0$ (with $\hat{e}_{\mu}$ the unit vector in direction $\mu$ and $a_0$ the lattice spacing) for the discrete derivative.  This discretization is not at all unique (Homework Excercise 1: find another discretization), but coarse physics like phase structure should not depend on the choice of discretization anyway.  

In Minkowski spacetime, the entities we try to compute from a quantum field theory are expressed in terms of functional integrals over all field configurations $\phi$, weighted by $\exp(i S[\phi])$.  E.g. the expectation of an operator $A(x,y,...)$ is given by
\[\left<A(x,y,...)\right> = \frac{\int \mathcal{D}\phi \: A(x_1,x_2,...) \exp(i S[\phi])}{\int \mathcal{D}\phi \: \exp(i S[\phi])}, \]
where $\int \mathcal{D}\phi$ means a functional integral over all configurations of $\phi$.  (This integral is tricky to define precisely for continuum fields, but when we discretize onto a lattice shortly it will become very well-defined.)  These are the oscillatory integrals alluded to earlier, with the oscillations coming from the $\exp(i S[\phi])$ term.  After Wick rotation this oscillatory term becomes a decaying exponential:
\[\frac{\int \mathcal{D}\phi \: A(x,y,...) \exp(-S_E[\phi])}{\int \mathcal{D}\phi \: \exp(-S_E[\phi])}.\]
Note that $x_0,y_0,...$ are rotated by $-i$ in this formula.  These Wick rotated integrals are ostensibly nicer, because of the exponential damping term.  They're still awefully big though, because we're integrating over an infinite dimensional space of fields.  Let's put everything on a lattice to salve our conscience:
\[ \left<A(x,y,...)\right> \rightarrow \frac{\prod_{z\in V(L)}\int d\phi(z) \: \phi(x)\phi(y) \exp(-S_E[\phi])}{\prod_{z\in V(L)}\int d\phi(z) \: \exp(-S_E[\phi])}\]
All we've done is replace $\int \mathcal{D}\phi$ with $\prod_{z\in V(L)}\int d\phi(z)$.  That is, integrating over all field configurations is now just integrating over the values of $\phi$ at each lattice site.  There are a finite number of lattice sites, so this is now a finite dimensional integral, which is something you can think of doing on a computer.  However, it's still still awefully big---the number of integrations is the number of lattice points, which can be thousands to millions or more for reasonably sized lattices.  Evaluating this efficiently is typically accomplished via Monte Carlo methods, which we will describe below.  For now, though, we finish the job at hand by identifying what we will want to compute when we get there.  The denominator of the above expression
\[Z := \prod_{z\in V(L)}\int d\phi(z) \: \exp(-S_E[\phi])\]
is recognized as essentially a partition function with Hamiltonian function $\mathcal{H}[\phi] := S_E[\phi]$ and inverse temperature $\beta=1$.  This looks worth computing.  It would probably be cooler if we could change the temperature $\beta$, so that it wasn't fixed at 1.  So lets redefine
\[Z(\beta) := \prod_{z\in V(L)}\int d\phi(z) \: \exp(-\beta S_E[\phi]).\]
The free energy is then defined in the usual way:
\begin{align*}
F(\beta)& := -\frac{d \log(Z)}{d\beta}\\
& = \frac{1}{Z}  \prod_{z\in V(L)}\int d\phi(z) \: S_E[\phi] \exp(-\beta S_E[\phi]).
\end{align*}
That is, the free energy is the expectation value of the action.  Free energy in this form is extrinsic, so to get a statistic that doesn't grow with lattice size we look at the free energy per site $f(\beta):=F(\beta)/N$, where $N$ is the number of lattice sites.  This is one of the basic quantities of interest to us in this paper.  The other is the heat capacity 
\begin{eqnarray*}
C(\beta)& := & \frac{-1}{N}\frac{df}{d\beta}\\
& = & \frac{1}{N^2}\frac{1}{Z}  \prod_{z\in V(L)}\int d\phi(z) \: S_E[\phi]^2 \exp(-\beta S_E[\phi]) \\
& & -\frac{1}{N^2}\frac{1}{Z^2} \left( \prod_{z\in V(L)}\int d\phi(z) \: S_E[\phi] \exp(-\beta S_E[\phi])\right)^2 \\
& = & \frac{1}{N^2}\left(\left<S_E^2\right> - \left<S_E\right>^2\right)
\end{eqnarray*}

The generalization to multiple fields, defined on either vertices or edges of $L$, is straightforward: Just replace the integration measure $\prod_{x \in V(L)} \int d\phi(x)$ by the product measure over all degrees of freedom, i.e. the field values at all edges or vertices.  For intrinsic quantities, divide by the total number of degrees of freedom (instead of by $N$ as above).

\section{Lattice gauge theories}
We now introduce concretely the stars of our show: lattice gauge theories (LGTs).  The form of the LGTs we consider has its genesis in discretizations of the Yang-Mills continuum field theory.  I won't go through the discretization here---I'll just state the result.  (Homework Excercise 2: Discretize the standard model.)

An LGT is specified by four things:
\begin{enumerate}
\item The lattice $L$.
\item Some fields $\{u_i\}$ on the lattice.
\item A gauge group $\mathcal{G}$ (with local symmetry group $G$) along with a group action on the fields $\{u_i\}$.
\item An action $S[\{u_i\}]$ and some observables $\{A[\{u_i\}]\}$, which should be invariant under gauge transformations.
\end{enumerate}

Our LGTs will consist of a single field $u(e)$ which lives on edges $E(L)$ of the lattice and takes values in the local symmetry group $G$.  The gauge group is the set of all $G$-valued fields $T$ on the \textit{vertices} $V(L)$.  The group action of a gauge transformation $T\in\mathcal{G}$ on $u$ is given by 
\begin{equation}
(T\cdot u)(e) = T(e_-) u(e) T(e_+)^{-1}
\label{gaugeAction}
\end{equation}
for any edge $e\in E(L)$.  This describes items (1)-(3) of the above list.  To describe the form of the last item---the action and observables---we must foray into some more detailed group theoretic considerations.  This is important enough it deserves its own subsection:

\subsection{Gauge invariance and group theory}
The action and observables of our LGTs must be functions of $u$ which are invariant under gauge transformations.  Now, looking at the form (\ref{gaugeAction}) of the gauge group action given above, we see that gauge transformations can make the value of $u$ at a given edge $e$ be anything whatsoever---for example, taking $T(e_+)=u(e)$ makes the transformed value $(T\cdot u)(e) = T(e_-) u(e) u(e)^{-1} = T(e_-)$, which is totally arbitrary.  So no function of $u$ at a single point in the lattice will be gauge invariant (except boring constant functions).  

To see what kinds of functions of $u$ \textit{are} gauge invariant, consider a product of the form $u(e^1)u(e^2)\cdots u(e^k)$, where the $e^j$ are edges satisfying $e_+^j=e_-^{j+1}$ for all $j\in\{1,2,...,k-1\}$.  That is, the edges form a \textit{path} of length k.  Under a gauge transformation $T$, we have 
\begin{multline*}
u(e^1)u(e^2)\cdots u(e^k) \rightarrow  \\
T(e_-^1)u(e^1)T(e_+^1)^{-1} T(e_-^2)u(e^2)T(e_+^2)^{-1} \cdots T(e_-^k)u(e^k)T(e_+^k)^{-1} \\ 
=  T(e_-^1)u(e^1)u(e^2)\cdots u(e^k)T(e_+^k). 
\end{multline*}
The factors of $T(\cdot)$ in the middle of the transformed product all cancel, leaving just the terms for the endpoints of the path.  So such a product along a path is \textit{almost} gauge invariant!  However, note the above requirement $e_+^j=e_-^{j+1}$ restricts us to paths that only move from lower vertices to upper vertices (what we will call the ``forward'' direction) of each edge along the way.  In general we would like to be able to traverse an edge in either direction, forward or backward.  Such a general path we will write as $(\pm_1 e^1, \pm_2 e^2, \dots , \pm_k e^k)$, where a $+e^j$ means edge $e^j$ is traversed forward and a $-e^j$ means it is traversed backward (the subscripts on the $\pm$ signs are just labels to differentiate them).  For this to give a well-defined path we must have $e^j_{\pm_j} = e^{j+1}_{\mp_{j+1}}$ (where $\mp_j$ indicates the sign opposite to $\pm_j$).  To make an \textit{almost} gauge invariant quantity out of the values $u(e^j)$ along this path, we construct the product 
\[P(\pm_1 e^1, \pm_2 e^2, \dots , \pm_k e^k) := u(e^1)^{\pm_1 1} u(e^2)^{\pm_2 1} \cdots u(e^k)^{\pm_k 1}.\]
For example 
\[P(-e^1,+e^2,\dots,+e^k) = u(e^1)^{-1} u(e^2)\cdots u(e^k).\]
I claim that this product is almost gauge invariant, in the same sense as before:
\begin{multline*}
P(\pm_1 e^1,\pm_2 e^2,\dots,\pm_k e^k) \rightarrow \\
\left(T(e_-^1)u(e^1)T(e_+^1)^{-1}\right)^{\pm_1} \cdots \left(T(e_-^k)u(e^k)T(e_+^k)^{-1}\right)^{\pm_k} \\
= T(e_{\mp_1}^1) u(e^1) u(e^2) \cdots u(e^k) T(e_{\pm_k}^k)^{-1} \\
= T(e_{\mp_1}^1) P(\pm_1 e^1,\pm_2 e^2,\dots,\pm_k e^k) T(e_{\pm_k}^k)^{-1}
\end{multline*}
The requirement $e^j_{\pm_j} = e^{j+1}_{\mp_{j+1}}$ produces the desired cancellation of most of the factors $T(\cdot)$.  

Now there's some bad news: For a path $(\pm_1 e^1, \dots , \pm_k e^k)$ which starts and ends at different points, the value $P(\pm_1 e^1, \dots , \pm_k e^k)$ can still be made to have arbitrary value by gauge transforming, using the same argument as for the case of a single edge.  However, if the path starts and ends at the same point---i.e. it is a loop---then $e_{\mp_1}^1 = e_{\pm_k}^k =: v_0$, and the product transforms as 
\begin{multline*}
P(\pm_1 e^1,\dots,\pm_k e^k) \rightarrow \\
= T(v_0) P(\pm_1 e^1,\dots,\pm_k e^k) T(v_0)^{-1}
\end{multline*}

This transformation is precisely \textit{conjugation by $T(v_0)$}.  This conjugation operation does not allow the value of the product to be completely arbitrary.  For example, if the group $G$ is Abelian, then $T(v_0) P(\pm_1 e^1,\dots,\pm_k e^k) T(v_0)^{-1} = P(\pm_1 e^1,\dots,\pm_k e^k)$, so the product is gauge invariant!  Even if $G$ is not Abelian, the values that the product can be transformed to are restricted to lie in the same \textit{conjugacy class}.  (The conjugacy class of an element $g\in G$ is the set $Cl(g):=\{hgh^{-1} : h\in G\}$.  It's not hard to show that these classes define an equivalence relation on $G$ and are invariant under conjugation by any element $h\in G$.  So in particular, conjugation by $T(v_0)$ does not change the conjugacy class of $ P(\pm_1 e^1,\dots,\pm_k e^k)$.)  This means the conjugacy class $Cl( P(\pm_1 e^1,\dots,\pm_k e^k) )$ is gauge invariant. 

In addition to gauge invariance, the conjugacy class $Cl( P(\pm_1 e^1,\dots,\pm_k e^k) )$ also has the nice property of being invariant under cyclic permutations of the edges,
\begin{eqnarray*}
& & Cl\left( P(\pm_2 e^2,\dots,\pm_k e^k,\pm_1 e^1)\right) \\
& = & Cl\left(  u(e^2)^{\pm_2 1} \cdots u(e^k)^{\pm_k 1} u(e^1)^{\pm_1 1} \right) \\
& = & u(e^1)^{\pm_1 1} Cl\left( u(e^2)^{\pm_2 1} \cdots u(e^1)^{\pm_1 1} \right)  u(e^1)^{\mp_1 1} \\
& = & Cl\left(  u(e^1)^{\pm_1 1} u(e^2)^{\pm_2 1} \cdots u(e^k)^{\pm_k 1} \right) \\
& = & Cl\left( P(\pm_1 e^1,\pm_2 e^2,\dots,\pm_k e^k)\right). \\
\end{eqnarray*}
So the conjugacy class is a property of the loop itself, independent of the order in which we parametrize it.  We will thus sometimes abbreviate our notation and write $Cl(u,l)$ for the conjugacy class of the above product of $u$ around the edges of a loop $l$.  

We are now in a position to describe the general form of a gauge invariant action (or observable) on fields $u$:  For any loop $l$, a \textit{loop operator} $A_l[u]$ is any function of $Cl(u,l)$.  Since $Cl(u,l)$ is gauge invariant, so is $A_l[u]$.  Furthermore, any composite function of multilple loop operators---e.g. the sum $A_{l_1}[u] + A_{l_2}[u]$ of two different loop operators---is also gauge invariant.  Such composite functions are what we will take as our operators and action.  It's that simple!

Having spelled out the general form of a gauge invariant action, we will now immediately specialize to the particular form of action which is most important.  (By ``most important'' I mean ``what everybody else is using''.)  The action $S[u]$ we will consider from here on is given by a sum over all \textit{plaquettes} $P$ (that is, square loops with one edge per side) of some real-valued function $s(Cl[u,P])$, where $s$ is an arbitrary real-valued function which I will call the \textit{plaquette action}.  The arbitrariness of $s$ means that this form of action is still quite broad.  When we start doing computations, we will explore different possibilities for the plaquette action.  However, it is also nice to have a certain standard form of plaquette action that we can use as a baseline, allowing us to make comparisons between different local symmetry groups $G$, for example.  To this end, define a baseline plaquette action $\delta(\cdot)$ by 

\begin{equation}
\delta(x) = 
\begin{cases}
0 & x = Cl(\mathbbm{1}_G) \\
1 & x \neq Cl(\mathbbm{1}_G) \\
\end{cases}
\end{equation}
where $\mathbbm{1}_G$ is the identity element of $G$.  Since all groups have an identity element, this baseline action always makes sense.  (Actually, the situation is even rosier than that: For any group $G$, the conjugacy class of the identity element consists of just the identity element itself.  In symbols, $Cl(\mathbbm{1}_G) = \{h\mathbbm{1}_G h^{-1} : h\in G\} = \{\mathbbm{1}_G\}$.  So this baseline action is very similar from one group to another.)  Physically, the interpretation of this action is that a plaquette is in its ``ground state'' when the product of the edges is the identity element, and any departure from this costs a uniform amount of energy.

To be totally concrete, the LGT action arising from this baseline plaquette action is 
\[S[u] = \sum_P \delta\left(Cl(u,P)\right),\]
where the sum is over all plaquettes.  With this action in hand we now have a fully specified lattice gauge theory, for any given gauge group $G$.  

Let me dwell on this last remark a bit: We have just constructed a lattice gauge theory for totally arbitrary local symmetry group $G$.  In particular, $G$ may be discrete.  This is in stark contrast to continuum gauge theories, where local symmetry groups are always continuous.  This is because of the requirement that gauge transformations be smooth, and thus continuous.  Any continuous function of a connected space (such as $\mathbb{R}^n$) into a discrete space must be constant, so a gauge transformation coming from a discrete local symmetry group must be constant, i.e. a global symmetry.  Gauge transformations on a lattice do not have this restriction, because a lattices are not connected.  This means we have lots of gauge theories on lattices that have no analog in the continuum.  These are the sorts of LGTs we will explore below. 

\section{Monte Carlo techniques}
Typical lattice gauge theory simulations are done on lattices having $\sim 10^4$ vertices, and comparable numbers of edges (to be precise, for a rectangular lattice, the number of edges equals the number of vertices times the dimension of the lattice).  For the smallest local symmetry group imaginable, $\mathbb{Z}_2$, this means the number of field configurations to sum over is $\sim 2^{10^4}\sim 10^{3010}\sim$ googol$^{30}$.  This is a bigger sum than most people can do in their head, even with a good night's sleep.  This is a bigger sum than most people can do on their laptop, even with Python.  This is a bigger sum than most people can do with a supercomputer, even with a very generous grant.  

So we have to come up with something else.  The something else is Monte Carlo simulation.  The basic idea is to sum not over all states, but rather over a representative subset.  In particular, recall that all the sums we want to do contain a Boltzmann weight $\exp(-\beta S[u])$.  For states where $S[u]$ is large compared to $1/\beta$, the Boltzmann weight is tiny, and that state doesn't contribute much to the sum.  So we don't need to pay as much attention to states with large action.  Conversely, we need to care a lot about states with small action.  The key idea of Monte Carlo simulation is to pick more of the states with small action than large action.  This is done by selecting states probabalistically, with the probability of selecting a given state proportional to its Boltzmann weight $\exp(-\beta S[u])$.  It can be proven that as we pick more and more states in this way, the sum over these states (which I'll call the ``Monte Carlo sum'') converges in a nice way to the sum over \textit{all} states, which is precisely what we want.  Crucially, the Monte Carlo sum involves a manageable number of states, so this technique is one we can deploy usefully on a computer. 




%%%%%%%%%% Bibliography

\begin{thebibliography}{9}

\bibitem{Creutz} 
	M. Creutz, L. Jacobs, and C. Rebbi, 
	Monte Carlo study of Abelian lattice gauge theories, 
	Physical Review D 20.8 (1979): 1915.

\bibitem{DAdda}
	A. D’Adda and P. Provero,
	Two-Dimensional Gauge Theories of the Symmetric Group S n in the Large-n Limit
	Communications in mathematical physics 245.1 (2004): 1-25.

\bibitem{Billo}
	M. Billo, A. D'Adda, and P. Provero,
	Branched coverings and interacting matrix strings in two dimensions,
	Nuclear Physics B 616.3 (2001): 495-516.

%\bibitem{BeckRobins} M. Beck and S. Robins, \textit{Computing the Continuous Discretely}, second edition, Springer-Verlag, New York, 2015, \url{http://dx.doi.org/10.1007/978-0-387-46112-0}

\end{thebibliography}

\end{document}